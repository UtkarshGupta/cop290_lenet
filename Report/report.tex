\documentclass{article}
\usepackage[margin=0.5in]{geometry}
\usepackage{graphicx}
\usepackage{natbib}
\usepackage{url}

\title{COP 290: Accelerating Matrix Multiplication}
\date{2019-02-14}
\author{Utkarsh Gupta (2016ME20756) \\ Utsav Das (2016CH10090)}

\begin{document}
\pagenumbering{gobble}
\maketitle

\newpage
\pagenumbering{arabic}
    \section{Overview of implemented libraries}
        \subsection{Convolution (Discrete)}
            $$ (f * g)[n] = \sum_{m=-\infty}^{\infty} f[n - m] * g[m] $$

        \subsection{Subsampling}
            Downsample the matrices to save computation time and trainable parameters requirement
            \subsubsection{Max Pooling}
                Use the maximum of a subsample to downsample
            \subsubsection{Avg Pooling}
                Use the average of a subsample to downsample

        \subsection{Non Linear Activation}
            \subsubsection{ReLU: Rectified Linear Unit}
                $$ f(x) = max(0, x) $$
            \subsubsection{tanh}
                $$ f(x) = tanh(x) $$

        \subsection{Vector Probabilities}
            \subsubsection{softmax}
                $$ f(x_i) = \frac{exp(x_i)}{\sum_{j=0}^{k} e^{-x_i}} $$
            \subsubsection{sigmoid}
                $$ f(x_i) = \frac{\mathrm{1}}{\mathrm{1} + e^{-x_i}} $$

\newpage
    \section{Convolution as matrix multiplication}
        Toeplitz matrix is constructed from the input matrix.
        Then the toeplitz matrix is multiplied with the filter in row major form.
        The rate determing step is the matrix-vector multiplication.

    \begin{figure}[!htb]
    \centering
        \includegraphics[width=6in]{old_gen_mm.eps}
        \caption{Matrix Multiplication (without optimization)}
    \end{figure}
    \begin{figure}[!htb]
    \centering
        \includegraphics[width=6in]{old_gen_conv.eps}
        \caption{Convolution (without optimization)}
    \end{figure}

\newpage
    \subsection{Accelerating matrix multiplication}
        To calculate the speeds for different convolution modes,
        we pass an additional argument "gen" to generate a matrix with all elements
        equal to a randomly generated float between 0 and 255. \\
        \subsubsection{Using OpenBLAS\cite{openblas} and Intel MKL\cite{intelmkl}}
            We'll be using cblas\_sgemv can use libraries such as openblas and intelmkl for matrix-vector multiplication.

    \begin{figure}[!htb]
    \centering
        \includegraphics[width=6in]{old_gen_mm_openblas.eps}
        \caption{Matrix Multiplication (with openblas)}
    \end{figure}
    \begin{figure}[!htb]
    \centering
        \includegraphics[width=6in]{old_gen_mm_intelmkl.eps}
        \caption{Matrix Multiplication (with intelmkl)}
    \end{figure}

\newpage
    \subsubsection{Using pthread}
        We can also use pthread to run our matrix-vector multiplication parallely.
    \begin{figure}[!htb]
    \centering
        \includegraphics[width=6in]{old_gen_mm_pthread.eps}
        \caption{Matrix Multiplication (with pthread)}
    \end{figure}
    \begin{figure}[!htb]
    \centering
        \includegraphics[width=6in]{old_gen_conv_pthread.eps}
        \caption{Convolution (with pthread)}
    \end{figure}

\newpage
    \section{Further optimization using g++ flags}
        g++ provides many compilation flags which significantly increase the performance.
        Flags used by us: "-Ofast -flto -fuse-linker-plugin -funroll-loops"

    \begin{figure}[!htb]
    \centering
        \includegraphics[width=6in]{gen_mm.eps}
        \caption{Matrix Multiplication (without optimization)}
    \end{figure}
    \begin{figure}[!htb]
    \centering
        \includegraphics[width=6in]{gen_conv.eps}
        \caption{Convolution (without optimization)}
    \end{figure}

    \begin{figure}[!htb]
    \centering
        \includegraphics[width=6in]{gen_mm_pthread.eps}
        \caption{Matrix Multiplication (with pthread)}
    \end{figure}
    \begin{figure}[!htb]
    \centering
        \includegraphics[width=6in]{gen_conv_pthread.eps}
        \caption{Convolution (with pthread)}
    \end{figure}

    \begin{figure}[!htb]
    \centering
        \includegraphics[width=6in]{gen_mm_openblas.eps}
        \caption{Matrix Multiplication (with openblas)}
    \end{figure}
    \begin{figure}[!htb]
    \centering
        \includegraphics[width=6in]{gen_mm_intelmkl.eps}
        \caption{Matrix Multiplication (with intelmkl)}
    \end{figure}

\clearpage
\bibliography{report}
\bibliographystyle{plain}
\end{document}
